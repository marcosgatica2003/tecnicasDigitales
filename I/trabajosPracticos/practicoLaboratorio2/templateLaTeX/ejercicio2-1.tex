\section{EJERCICIO 2.1: Utilizar conversor BCD}\sangria{} La consigna de este ejercicio es utilizar un conversor BCD a 7 segmentos CD4511 con el minilab.\subsection{Materiales usados}\begin{itemize}[nosep]\item Mini laboratorio (minilab) \item Decodificador BCD a 7 segmentos CD4511 \item Resistores \item Display de 7 segmentos \item Fuente de alimentación $V_{CC}$ \item Cables Dupont\end{itemize}\subsection{Procedimiento}\begin{enumerate}[nosep]\item Análisis de datasheet del CD4511.\item Armado de esquemático.\item Armado de circuito. \item Identificado de pines del decodificador BCD a 7 segmentos y el display de 7 segmentos según fabricante. \item Agregado de resistores limitadores de corrientes. \item Alimentación de circuito. \item Probado con una entrada en formato BCD de 4 bits al decodificador y visualización del número correspondiente en el display de 7 segmentos. \item Comprobación de diferentes entradas BCD y su salida en el display de 7 segmentos. \end{enumerate}\imagen[Implementación]{10cm}{./imagenes/implementacion0.jpg} \subsection{Preguntas de análisis}\begin{enumerate}\item \textbf{¿Cuál es la función del decodificador BCD a 7 segmentos?}\\ $\rightarrow$ El decodificador CD4511 es un circuito integrado BCD (decimal codificado a binario) a siete segmentos que permite mostrar valores numéricos en un display usando de entrada un código BCD. Sirve como ''traductor'' de instrucciones (combinaciones binarias de 4 bits 0000-1001) en las señales adecuadas para activar segmentos del display, de modo que se muestre el digito correspondiente del 0 al 9. \item \textbf{¿Cuál es la conexión adecuada entre el decodificador y el display de 7segmentos?}\\ $\rightarrow$ El integrado CD4511 posee 4 entradas de datos: \textbf{D C B A} para el código en BCD. Posee siete salidas: $a \rightarrow g$ que controlan los segmentos del display. Cuenta con entradas de control (LATCH ENABLE), BLANKING y LAMP TEST que permiten almacenar de forma tempotal un valor, apagar el display sin perder el dato o encender todos los segmentos simultáneamente. \imagen[Distribución de pines CD4511]{7cm}{./imagenes/pinesCD4511.png} \item \textbf{¿Qué sucede si se proporciona una entrada inválida al decodificador?}\\ $\rightarrow$ Dicho escenario se responde mirando la tabla de verdad del decodificador CD4511: \imagen[Tabla de verdad CD4511]{8cm}{./imagenes/tablaVerdadCD4511.png} Cuando se ingresa un valor binario existente en la tabla, se activan las salidas correspondientes para formar el digito en el display, para cualquier otro valor incorrecto (ej: mayor que 1001), el decodificador apaga el display poniendo sus salidas a 0 (los ''Blank''). 
%\item \textbf{¿Cuál es la relación entre los bits de entrada y los segmentos del display?} \item \textbf{¿Cuál es la utilidad de los resistores del circuito?} 
\end{enumerate}
