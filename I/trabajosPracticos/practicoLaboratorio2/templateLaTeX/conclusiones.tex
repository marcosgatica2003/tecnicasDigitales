\newpage
\section{Conclusiones}\sangria{}En el trabajo pudimos observar de manera notable como afectan los distintos voltajes a la corriente que pasa por el canal. Queda visible en el desarrollo del trabajo como a medida que se aumenta el voltaje entre compuerta y surtidor disminuye la corriente y como el voltaje entre drenador y surtidor no modifica la corriente del canal cuando ésta llega a la zona activa. Pudimos entender el funcionamiento del JFET pero tuvimos demasiadas dificultades a la hora de implementar las consignas ya que no conseguimos un buen modelo de JFET, ni en la vida real ni en la simulación, pero comparando con compañeros discutimos los resultados y concluimos que fueron medianamente acertados.
\sangria{}Analizando los resultados de las prácticas y comparandolos con la teoría, podemos decir que hicimos una buenas mediciones.
\sangria{}Los problemas del transistor se deben a que conseguimos uno que tenía estática entre la Gate y la Source, entonces al tocarlo rompimos varias veces.
\sangria{}Para concluir podemos decir que hicimos bien las actividades de laboratorio con los recursos que teniamos y pudimos sortear bien las dificultades.
