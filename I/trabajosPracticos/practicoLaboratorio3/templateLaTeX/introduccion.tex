\section{RÚBRICA}

\begin{center}

\begin{tabular}{|l|l|l|}
\hline
\rowcolor[HTML]{FFFFC7} 
\multicolumn{1}{|c|}{\cellcolor[HTML]{FFFFC7}\textbf{TAREA}}        & \multicolumn{1}{c|}{\cellcolor[HTML]{FFFFC7}\textbf{PUNTUACIÓN}} & \multicolumn{1}{c|}{\cellcolor[HTML]{FFFFC7}\textbf{MÁXIMO}} \\ \hline
Presentación del informe con archivos correspondientes              &                                                                  & $20\%$                                                       \\ \hline
Comprender la estructura interna y funcionamiento del inversor CMOS &                                                                  & $30\%$                                                       \\ \hline
Comprender las etapas involucradas en el flujo de diseño analógico  &                                                                  & $30\%$                                                       \\ \hline
Defensa de las conclusiones                                         &                                                                  & $20\%$                                                       \\ \hline
\cellcolor[HTML]{FFFFC7}\textbf{TOTAL}                              &                                                                  & $100\%$                                                      \\ \hline
\end{tabular}

\end{center}

\newpage

\section{INTRODUCCIÓN}
\sangria{} El objetivo de este informe es comprender y aplicar el flujo de diseño analógico para circuitos integrados en tecnología CMOS, así como conocer y utilizar las herramientas \textit{OpenSource} de diseño de C.I. para validar esquemáticos y layouts. Para ello nos valimos de familiarizarse con las reglas de diseño físico (DRC), simulaciones y verificaciones par agarantizar la compatibilidad con los procesos de fabricación, e integrar el conocimiento teórico con la práctica utilizando un kit de diseño: (PDK) SKY130, junto a herramientas CAD para un diseño confiable y escalable.

\subsection{Condiciones de trabajo y ensayo}
\sangria{} El sistema operativo usado en este trabajo fue Gentoo GNU/Linux. Elegido por simple comodidad.
\imagen[Logo Gentoo]{3cm}{./imagenes/logoGentoo.png}
\subsection{Consigna del TP: Especificaciones de diseño}
\sangria{} El inversor CMOS está diseñado para cumplir:\\[3pt]
\begin{itemize}[nosep]
    \item Tecnología SKY130 ($130$ $nm$ $CMOS$)
    \item Voltaje de alimentación $V_{DD} = 1,8V$
    \item Especificaciones de diseño para PMOS:
        \begin{itemize}[nosep]
            \item $L = 0,15$ (Longitud del canal del transistor en micrómetros).
            \item $W_p = 2,1$ (Ancho del canal del transistor en micrómetros).
            \item $nf = 1$ (Número de fingers).
            \item $mult = 1$ (Multiplicidad).
            \item model: $pfet_01v8$
        \end{itemize}
    \item Especificaciones de diseño para NMOS:
        \begin{itemize}[nosep]
            \item $L = 0,15$ (Longitud del canal del transistor en micrómetros).
            \item $W_p = 1,05$ (Ancho del canal del transistor en micrómetros).
            \item $nf = 1$ (Número de fingers).
            \item $mult = 1$ (Multiplicidad).
            \item model: $nfet_01v8$
        \end{itemize}
    \item Tiempo de transición objetivo: $<50 ns$ (subida y bajada)
\end{itemize}
