\newcommand{\implicantsol}[3][0]{
    \draw[rounded corners=3pt, fill=#3, opacity=0.3] ($(#2.north west)+(135:#1)$) rectangle ($(#2.south east)+(-45:#1)$);
    }
\newcommand{\implicant}[4][0]{
    \draw[rounded corners=3pt, fill=#4, opacity=0.3] ($(#2.north west)+(135:#1)$) rectangle ($(#3.south east)+(-45:#1)$);
    }
\newcommand{\implicantcostats}[4][0]{
    \draw[rounded corners=3pt, fill=#4, opacity=0.3] ($(rf.east |- #2.north)+(90:#1)$)-| ($(#2.east)+(0:#1)$) |- ($(rf.east |- #3.south)+(-90:#1)$);
    \draw[rounded corners=3pt, fill=#4, opacity=0.3] ($(cf.west |- #2.north)+(90:#1)$) -| ($(#3.west)+(180:#1)$) |- ($(cf.west |- #3.south)+(-90:#1)$);
}
\newcommand{\inicioCodigo}{
    \newpage
    \fancyfoot[C]{}
    \onecolumn
}

\newcommand{\finalCodigo}{
    \newpage
    \fancyfoot[C]{
        \begin{tikzpicture}[remember picture, overlay]
            \draw[gray!30, line width=0.4pt]
                (0cm, 1cm) -- (0cm, 25cm);
        \end{tikzpicture}
    }
    \twocolumn
}


\newcommand{\implicantdaltbaix}[4][0]{
    \draw[rounded corners=3pt, fill=#4, opacity=0.3] ($(cf.south -| #2.west)+(180:#1)$) |- ($(#2.south)+(-90:#1)$) -| ($(cf.south -| #3.east)+(0:#1)$);
    \draw[rounded corners=3pt, fill=#4, opacity=0.3] ($(rf.north -| #2.west)+(180:#1)$) |- ($(#3.north)+(90:#1)$) -| ($(rf.north -| #3.east)+(0:#1)$);
}

\newcommand{\implicantcantons}[2][0]{
    \draw[rounded corners=3pt, opacity=.3] ($(rf.east |- 0.south)+(-90:#1)$) -| ($(0.east |- cf.south)+(0:#1)$);
    \draw[rounded corners=3pt, opacity=.3] ($(rf.east |- 8.north)+(90:#1)$) -| ($(8.east |- rf.north)+(0:#1)$);
    \draw[rounded corners=3pt, opacity=.3] ($(cf.west |- 2.south)+(-90:#1)$) -| ($(2.west |- cf.south)+(180:#1)$);
    \draw[rounded corners=3pt, opacity=.3] ($(cf.west |- 10.north)+(90:#1)$) -| ($(10.west |- rf.north)+(180:#1)$);
    \fill[rounded corners=3pt, fill=#2, opacity=.3] ($(rf.east |- 0.south)+(-90:#1)$) -|  ($(0.east |- cf.south)+(0:#1)$) [sharp corners] ($(rf.east |- 0.south)+(-90:#1)$) |-  ($(0.east |- cf.south)+(0:#1)$) ;
    \fill[rounded corners=3pt, fill=#2, opacity=.3] ($(rf.east |- 8.north)+(90:#1)$) -| ($(8.east |- rf.north)+(0:#1)$) [sharp corners] ($(rf.east |- 8.north)+(90:#1)$) |- ($(8.east |- rf.north)+(0:#1)$) ;
    \fill[rounded corners=3pt, fill=#2, opacity=.3] ($(cf.west |- 2.south)+(-90:#1)$) -| ($(2.west |- cf.south)+(180:#1)$) [sharp corners]($(cf.west |- 2.south)+(-90:#1)$) |- ($(2.west |- cf.south)+(180:#1)$) ;
    \fill[rounded corners=3pt, fill=#2, opacity=.3] ($(cf.west |- 10.north)+(90:#1)$) -| ($(10.west |- rf.north)+(180:#1)$) [sharp corners] ($(cf.west |- 10.north)+(90:#1)$) |- ($(10.west |- rf.north)+(180:#1)$) ;
}

\newenvironment{Karnaugh}%
{
\begin{tikzpicture}[baseline=(current bounding box.north),scale=0.8]
\draw (0,0) grid (4,4);
\draw (0,4) -- node [pos=0.7,above right,anchor=south west] {cd} node [pos=0.7,below left,anchor=north east] {ab} ++(135:1);
%
\matrix (mapa) [matrix of nodes,
        column sep={0.8cm,between origins},
        row sep={0.8cm,between origins},
        every node/.style={minimum size=0.3mm},
        anchor=8.center,
        ampersand replacement=\&] at (0.5,0.5)
{
                       \& |(c00)| 00         \& |(c01)| 01         \& |(c11)| 11         \& |(c10)| 10         \& |(cf)| \phantom{00} \\
|(r00)| 00             \& |(0)|  \phantom{0} \& |(1)|  \phantom{0} \& |(3)|  \phantom{0} \& |(2)|  \phantom{0} \&                     \\
|(r01)| 01             \& |(4)|  \phantom{0} \& |(5)|  \phantom{0} \& |(7)|  \phantom{0} \& |(6)|  \phantom{0} \&                     \\
|(r11)| 11             \& |(12)| \phantom{0} \& |(13)| \phantom{0} \& |(15)| \phantom{0} \& |(14)| \phantom{0} \&                     \\
|(r10)| 10             \& |(8)|  \phantom{0} \& |(9)|  \phantom{0} \& |(11)| \phantom{0} \& |(10)| \phantom{0} \&                     \\
|(rf) | \phantom{00}   \&                    \&                    \&                    \&                    \&                     \\
};
}%
{
\end{tikzpicture}
}

%Empty Karnaugh map 2x4
\newenvironment{Karnaughvuit}%
{
\begin{tikzpicture}[baseline=(current bounding box.north),scale=0.8]
\draw (0,0) grid (4,2);
\draw (0,2) -- node [pos=0.7,above right,anchor=south west] {bc} node [pos=0.7,below left,anchor=north east] {a} ++(135:1);
%
\matrix (mapa) [matrix of nodes,
        column sep={0.8cm,between origins},
        row sep={0.8cm,between origins},
        every node/.style={minimum size=0.3mm},
        anchor=4.center,
        ampersand replacement=\&] at (0.5,0.5)
{
                      \& |(c00)| 00         \& |(c01)| 01         \& |(c11)| 11         \& |(c10)| 10         \& |(cf)| \phantom{00} \\
|(r00)| 0             \& |(0)|  \phantom{0} \& |(1)|  \phantom{0} \& |(3)|  \phantom{0} \& |(2)|  \phantom{0} \&                     \\
|(r01)| 1             \& |(4)|  \phantom{0} \& |(5)|  \phantom{0} \& |(7)|  \phantom{0} \& |(6)|  \phantom{0} \&                     \\
|(rf) | \phantom{00}  \&                    \&                    \&                    \&                    \&                     \\
};
}%
{
\end{tikzpicture}
}

%Empty Karnaugh map 2x2
\newenvironment{Karnaughquatre}%
{
\begin{tikzpicture}[baseline=(current bounding box.north),scale=0.8]
\draw (0,0) grid (2,2);
\draw (0,2) -- node [pos=0.7,above right,anchor=south west] {b} node [pos=0.7,below left,anchor=north east] {a} ++(135:1);
%
\matrix (mapa) [matrix of nodes,
        column sep={0.8cm,between origins},
        row sep={0.8cm,between origins},
        every node/.style={minimum size=0.3mm},
        anchor=2.center,
        ampersand replacement=\&] at (0.5,0.5)
{
          \& |(c00)| 0          \& |(c01)| 1  \\
|(r00)| 0 \& |(0)|  \phantom{0} \& |(1)|  \phantom{0} \\
|(r01)| 1 \& |(2)|  \phantom{0} \& |(3)|  \phantom{0} \\
};
}%
{
\end{tikzpicture}
}

%Defines 8 or 16 values (0,1,X)
\newcommand{\contingut}[1]{%
\foreach \x [count=\xi from 0]  in {#1}
     \path (\xi) node {\x};
}

%Places 1 in listed positions
\newcommand{\minterms}[1]{%
    \foreach \x in {#1}
        \path (\x) node {1};
}

%Places 0 in listed positions
\newcommand{\maxterms}[1]{%
    \foreach \x in {#1}
        \path (\x) node {0};
}

%Places X in listed positions
\newcommand{\indeterminats}[1]{%
    \foreach \x in {#1}
        \path (\x) node {X};
}


\renewcommand{\theenumi}{\Roman{enumi}}
\definecolor{gold}{rgb}{1.0, 0.84, 0.0}
\newcommand{\resistencia}[4]{
    \begin{minipage}{2cm}
        \begin{center}
            \begin{tikzpicture}
                \draw(-0.4cm, -0.15cm) rectangle (0.4cm, 0.15cm);
                \fill[resistor] (-0.4cm, 0) -- (-0.5cm, 0);
                \draw (-0.4cm, 0) -- (-0.5cm, 0);
                \draw (0.4cm, 0) -- (0.5cm, 0);
                \fill[#1] (-0.35cm, -0.15cm) rectangle (-0.25cm, 0.15cm);
                \fill[#2] (-0.15cm, -0.15cm) rectangle (-0.05cm, 0.15cm);
                \fill[#3] (0.05cm, -0.15cm) rectangle (0.15cm, 0.15cm);
                \fill[#4] (0.25cm, -0.15cm) rectangle (0.35cm, 0.15cm);
            \end{tikzpicture}
        \end{center}
    \end{minipage}
}
\newcommand{\diodoSilicio}[1]{
    \begin{minipage}{2cm}
        \begin{center}
            \begin{circuitikz}[scale=0.5]
                \draw[line width=1pt] (-2,0) -- (-1,0);
                \draw[line width=1pt] (2,0) -- (1,0);

                \filldraw[fill=black!75, draw=black] (-1, -0.5) rectangle (1, 0.5);

                \fill[gray!60] (0.6, -0.5) rectangle (0.8, 0.5);

                \node at (0, -1) {\texttt{#1}};
            \end{circuitikz}
        \end{center}
    \end{minipage}
}
\newcommand{\diodoGermanio}[1]{
    \begin{minipage}{2cm} 
        \begin{center}
            \begin{circuitikz}[scale=0.5]
                \draw[line width=0.8pt] (-2.5,0) -- (-1.5,0);
                \draw[line width=0.8pt] (2.5,0) -- (1.5,0);

                \filldraw[fill=white, draw=black] (-1.5, -0.4) rectangle (1.5, 0.4);
                \fill[orange!80] (-1.5, -0.25) rectangle (1.5, 0.25);
                \fill[gray!95] (1.2, -0.4) rectangle (1.5, 0.4);
                \node at (0, -0.9) {\texttt{#1}};
            \end{circuitikz}
        \end{center}
    \end{minipage}
}
\newcommand{\inv}[1]{\frac{1}{#1}}
\newcommand{\recuadrar}[2]{
    \begin{center}
        \begin{tabular}{|m{#1}|}
            \toprule
            \multicolumn{1}{|c|}{\hspace{5pt}#2\hspace{5pt}} \\
            \bottomrule
        \end{tabular}
    \end{center}
}

\newcommand{\ecuacion}[1]{
    \begin{center}
        \[
            #1
        \]
    \end{center}
}

\newcommand{\saltoPag}[0]{%
    \newpage\noindent\thispagestyle{fancy}%
    \begin{tikzpicture}[remember picture, overlay]
        \draw[line width=0.4pt, color=gray!50] (8.5cm, -0.4cm) -- (8.5cm, -24cm);
    \end{tikzpicture}%
    \ignorespaces{}
}

\newcommand{\midTitle}[2]{%
    \begin{center}
        \textcolor{#1}{\underline{#2}} 
    \end{center}
}

\captionsetup{
    format=hang,
    labelfont={bf},
    textfont=normalfont,
    labelsep=colon,
    justification=centering,
    singlelinecheck=true
}

\newcommand{\imagen}[3][]{
    \begin{center}
        \begin{tikzpicture}
            \node[inner sep=2pt] (image) {\includegraphics[width=#2]{#3}};
            \draw[line width=1pt, color={rgb:red,251;green,73;blue,52}] (image.south west) rectangle (image.north east);
        \end{tikzpicture}
        \ifx\\#1\\
            \captionof{figure}{}
        \else
            \captionof{figure}{#1}
        \fi 
        \label{fig:#2}
    \end{center}
}

\newcommand{\sangria}[0]{\par\noindent\hspace*{15pt}}
